% !TEX root =  ../Dissertation.tex

\chapter{Deep-RL Implementation}
\section{Rainbow Agent}
implementation adapted from benJMiddleton\cite{BenJMiddleton}
introduction on what rainbow is: combination of many extensions to DQN - Include brief explanations\cite{hessel2018rainbow}

\begin{itemize}
    \item Double Q-learning \cite{van2016doubleq}
    \item Prioritized Replay \cite{schaul2015prioritized}
    \item Duelling Networks \cite{wang2016dueling}
    \item multi-step learning \cite{sutton2018reinforcement}
    \item Distributional RL\cite{bellemare2017distributional}
    \item Noisy Nets\cite{DBLP:journals/corr/FortunatoAPMOGM17}
\end{itemize}

\section{System Architecture}
How deep-rl interacts with dolphin. python libraries (pytorch etc. not supported by embedded python in dolphin) -> use ports to send json objects to and from dolphin process to rainbow process
\\ what rainbow does at each time step

\begin{figure}[ht]
    \centering
    \includegraphics{Preamble/BirmCrest.png}
    \caption{Diagram of D-RL Framework}
    \label{fig:rainbow-arch}
\end{figure}

\section{Input Processing}
how frame data is processed before going to rainbow
\begin{figure}[ht]
    \centering
    \includegraphics{Preamble/BirmCrest.png}
    \caption{Screenshot of game before and after processing}
    \label{fig:frame-processing}
\end{figure}
\\pixel data along with same info as q-learning
\\how does removing frameInfo data affect deep-learning performance - pixel vs frameInfo
\\how its processed within rainbow - cnn