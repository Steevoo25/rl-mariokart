\chapter*{Introduction}
\addcontentsline{toc}{chapter}{Introduction}
Games, despite primarily being recreational hobbies, can be used as useful environments to test and compare the effectiveness of new developments in AI. Mnih \textit{(et al.)} (2015) \cite{mnih2015human} were able to achieve a performance level comparable to that of a professional human games tester in the Atari Learning Environment (ALE) \cite{bellemare2013arcade} through an algorithm called DQN \cite{mnih2013playing}. These developments in AI, specifically RL, for games can lead to insights into other disciplines, informing decision making in a wide range of situations.
An example of an interesting and potentially life-changing discipline is self-driving cars, where this research has helped lead to successful demonstrations of end-to-end driving \cite{DBLP:journals/corr/BojarskiTDFFGJM16} and high-level decision making \cite{duan2020hierarchical}. However, these complex developments come at a cost. Image processing in particular is heavily GPU reliant and therefore not always accessible, particularly to those without this hardware. My project considers the effectiveness and relevance of 'traditional' RL approaches compared to D-RL when applied to the same scenario. I aim to achieve this by giving two formulations of my chosen environment, one for RL and one for D-RL, along with comparing the corresponding trained agents against each other and human performance. Through this, I investigate whether human level performance is also achievable through 'traditional' RL approaches. Specifically, I compare Q-Learning (RL) \cite{watkins1992q}, a model-free RL algorithm, and Rainbow (D-RL) \cite{hessel2018rainbow},a variation of DQN.
The scenario I have chosen for this is Mariokart Wii. The game is hugely popular, with over 38 million copies sold. This means that this is a familiar environment for many, unlocking the potential for an application in education. This, paired with my great interest and passion for the game, made it a clear choice for my project. 