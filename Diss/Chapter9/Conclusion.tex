% !TEX root =  ../Dissertation.tex
\chapter{Conclusions}
%conclude these nuts
In this paper I have shown implementations of two different approaches to Reinforcement Learning in a simplified Mariokart Wii environment. After tuning and training I found that my Q-learning implementation achieved performance similar to an average human player, demonstrating the effectiveness of my formulation. My deep learning approach was unsuccessful, however I was still able to compare the two using data from an existing implementation. Overall, I found both approaches were equivalent to a human player with intermediate skill level after less than 20 hours of training, with the potential to surpass this level with a few minor adjustments and additional training time. Similarly to Mnih \cite{mnih2015human}, I have shown that human level performance is possible through Reinforcement Learning. I then discussed some limitations with my project, such as its lack of generalisation to other tracks or vehicles and the inability to use items. Additionally, I proposed solutions to these limitations, mainly detailing various additions to the state space of my agent. Finally, I outlined a series of potential future investigations, such as including an enhanced action space with pre-programmed multi-step inputs.