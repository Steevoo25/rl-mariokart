% !TEX root =  ../Dissertation.tex

\chapter{Emulator Specification}
\section{motivations}
previous experience\\ 
most widely used\\
api access\\
savestates for resetting\\
dumping pixel data\\
allows multiple concurrent controllers - possible to play against agent\\

\section{Memory access}
\subsection{Process}
iterative search process using Dolphin Memory Engine + Cheat code to find required memory addresses
\\cheat code gave value to search for
\\ iteratively search through memory until address was found 
\\ use api to read addresses during execution
\section{Descriptions}
\begin{itemize}
    \item Speed - current speed of vehicle
    \item Race Completion \% - current completion of entire race 1-2 = lap 1, 2-3 = lap 2, 3-4 = lap 3, 4 = finished
    \item Miniturbo charge - 0-270
    \item Wheelie = 0,1
    \item Road Type = Normal road = 0, Offroad = 3
\end{itemize}
\begin{table}[h]
    \centering
    \begin{tabular}{l|l|c}
    \textbf{Name}  & \textbf{Location}\\
    \hline
     XZ Speed  &    0x0\\
     Race\%  &   0x0\\
     Miniturbo Charge &  0x0 \\
     Wheelie State &  0x0 \\
     Road Type &  0x0 \\
    \end{tabular}
    \caption{Memory Addresses in Mario Kart Wii's memory}
    \label{tab:memory-addresses}
\end{table}